\documentclass[12pt]{article}
\usepackage{amsmath}
\usepackage{amssymb}
\usepackage{geometry}
\usepackage{hyperref}
\usepackage{graphicx}
\usepackage{enumitem}
\geometry{margin=1in}

\title{Hierarchical Multi-Agent Architecture for Real-Time Codeforces Problem Solving with Chess-Inspired Strategic Abstraction}
\author{}
\date{}

\begin{document}

\maketitle

\section{Abstract}

This document proposes a hierarchical multi-agent system designed to solve Codeforces problems in real time while providing structured explanations inspired by chess strategic archetypes. The system integrates role-based agent hierarchy, session-scoped file memory, rating-based governance, and computational cost control. The objective is to improve reasoning robustness, reduce hallucination, and maintain cost efficiency.

\section{Motivation}

Competitive programming and chess share structural similarities:
\begin{itemize}
    \item Adversarial reasoning
    \item Strategic pattern recognition
    \item Resource optimization under constraints
    \item Evaluation of alternative branches
\end{itemize}

The proposed system models problem solving as an organizational hierarchy analogous to corporate structures and chess decision processes.

\section{System Overview}

The architecture consists of:

\begin{enumerate}
    \item Hierarchical agent roles
    \item Session-scoped structured memory
    \item Rating-based activation control
    \item External execution verification
    \item Token and computation cost control
\end{enumerate}

\section{Hierarchical Agent Roles}

The system uses role specialization instead of monolithic reasoning.

\subsection{Intern Agent}

Responsibilities:
\begin{itemize}
    \item Extract problem constraints
    \item Propose candidate paradigms
    \item Estimate rough time complexity
\end{itemize}

Output schema:

\begin{verbatim}
{
  "proposed_paradigm": "...",
  "estimated_complexity": "...",
  "confidence": 0.xx
}
\end{verbatim}

\subsection{Engineer Agent}

Responsibilities:
\begin{itemize}
    \item Refine candidate strategies
    \item Eliminate infeasible approaches
    \item Structure algorithm design
\end{itemize}

\subsection{Senior Engineer Agent}

Responsibilities:
\begin{itemize}
    \item Prove correctness
    \item Identify corner cases
    \item Optimize complexity
\end{itemize}

\subsection{Lead Agent}

Responsibilities:
\begin{itemize}
    \item Compare multiple viable strategies
    \item Select optimal trade-off
    \item Validate scalability
\end{itemize}

\subsection{CEO Agent}

Responsibilities:
\begin{itemize}
    \item Final approval
    \item Confidence calibration
    \item Rating update
    \item Generate final structured explanation
\end{itemize}

Authority flows upward. Feedback may flow across roles, but finalization authority remains hierarchical.

\section{Session-Scoped File Memory}

Each problem operates inside an isolated session workspace:

\begin{verbatim}
/session_id/
    problem_spec.json
    constraint_analysis.json
    candidate_strategies.json
    decision_log.json
    proof_notes.md
    test_results.json
    final_solution.cpp
\end{verbatim}

Key principles:

\begin{itemize}
    \item Memory is isolated per problem
    \item Files are structured and version-controlled
    \item Agents interact only via file updates
    \item No cross-session contamination
\end{itemize}

\section{Rating System}

Each problem has a rating $R_p$.

Each agent tier has a rating $R_a$.

Expected success probability:

\[
E = \frac{1}{1 + 10^{(R_p - R_a)/400}}
\]

Rating update rule:

\[
R_{new} = R_{old} + K(S - E)
\]

Where:
\begin{itemize}
    \item $S = 1$ if solved successfully
    \item $S = 0$ otherwise
    \item $K$ is update constant
\end{itemize}

Hierarchy activation depends on problem rating. Lower-rated problems do not activate full stack.

\section{Hallucination Mitigation}

The system reduces hallucination through:

\begin{enumerate}
    \item Structured JSON outputs
    \item Explicit constraint files
    \item External execution validation
    \item Limited revision cycles
    \item Confidence-based escalation
\end{enumerate}

Execution validation is mandatory. Logical claims are verified through runtime testing.

\section{Computation and Cost Control}

\subsection{Model Tiering}

\begin{itemize}
    \item Lower roles use smaller models
    \item Higher roles use larger models only when needed
\end{itemize}

\subsection{Token Budgeting}

Each role has a strict token limit. Outputs exceeding limit must be summarized.

\subsection{Adaptive Escalation}

Higher-tier agents activate only if:
\begin{itemize}
    \item Confidence is low
    \item Complexity risk detected
    \item Verification fails
\end{itemize}

\section{Comparative Strategy Evaluation}

Multiple strategies may be proposed. Comparison is based on:

\begin{itemize}
    \item Asymptotic complexity
    \item Empirical runtime
    \item Edge case robustness
    \item Confidence score
\end{itemize}

Only top-ranked strategy proceeds to final approval.

\section{Chess-Inspired Strategic Abstraction Layer}

The system optionally maps algorithm paradigms to chess archetypes.

Example mapping:

\begin{itemize}
    \item Greedy $\rightarrow$ Initiative play
    \item Dynamic Programming $\rightarrow$ Positional accumulation
    \item Backtracking $\rightarrow$ Variation calculation
    \item Game Theory $\rightarrow$ Perfect adversarial play
\end{itemize}

This layer enhances pedagogical clarity but does not affect correctness verification.

\section{Termination Criteria}

A session concludes when:

\begin{itemize}
    \item All test cases pass
    \item Confidence exceeds threshold
    \item Maximum revision cycles reached
\end{itemize}

Session memory is then archived or discarded.

\section{Conclusion}

The proposed system combines hierarchical reasoning, structured memory, rating governance, and execution validation to build a robust and cost-aware multi-agent solver for Codeforces problems. The design emphasizes convergence, verification, and computational efficiency while enabling optional strategic abstraction inspired by chess.

\section{Formal System Model}

\subsection{Problem Definition}

Let a Codeforces problem be defined as:

\[
P = (I, O, C, T)
\]

Where:
\begin{itemize}
    \item $I$ = Input specification
    \item $O$ = Output specification
    \item $C$ = Constraint set
    \item $T$ = Tag set (algorithmic hints)
\end{itemize}

The objective of the system is to construct:

\[
S = (A, \Pi, \Phi)
\]

Where:
\begin{itemize}
    \item $A$ = Algorithm
    \item $\Pi$ = Proof of correctness
    \item $\Phi$ = Complexity characterization
\end{itemize}

Such that $A$ satisfies $C$ and produces correct $O$ for all valid $I$.

\subsection{Hierarchical Agent Model}

Define a set of agents:

\[
\mathcal{H} = \{H_1, H_2, H_3, H_4, H_5\}
\]

Where:
\begin{itemize}
    \item $H_1$ = Intern
    \item $H_2$ = Engineer
    \item $H_3$ = Senior Engineer
    \item $H_4$ = Lead
    \item $H_5$ = CEO
\end{itemize}

Each agent $H_i$ performs a transformation:

\[
H_i : (P, M_i) \rightarrow (M_{i+1})
\]

Where $M_i$ represents the structured session memory at stage $i$.

Authority ordering is defined as:

\[
H_1 \prec H_2 \prec H_3 \prec H_4 \prec H_5
\]

Meaning decisions can only be finalized by higher-order agents.

\section{Session Memory State Machine}

Let the session state be:

\[
\Sigma = (F, D, V)
\]

Where:
\begin{itemize}
    \item $F$ = File state (problem, constraints, strategies)
    \item $D$ = Decision log
    \item $V$ = Verification status
\end{itemize}

State transitions follow:

\[
\Sigma_0 \rightarrow \Sigma_1 \rightarrow \dots \rightarrow \Sigma_k
\]

Termination condition:

\[
\Sigma_k \text{ is terminal if } V = \text{pass} \land \text{confidence} > \tau
\]

Where $\tau$ is confidence threshold.

\section{Computation Cost Model}

Let:

\[
C_{total} = \sum_{i=1}^{n} (t_i \cdot \lambda_i)
\]

Where:
\begin{itemize}
    \item $t_i$ = tokens used by agent $i$
    \item $\lambda_i$ = cost per token for model used by agent $i$
\end{itemize}

To maintain cost efficiency:

\[
t_i \leq T_i^{max}
\]

Where $T_i^{max}$ is strict token budget per role.

Escalation occurs only if:

\[
\text{confidence}_i < \theta_i
\]

Where $\theta_i$ is escalation threshold.

\section{Adaptive Activation Policy}

Let problem rating be $R_p$.

Define activation function:

\[
\alpha(R_p) =
\begin{cases}
\{H_1, H_2\} & R_p < 1200 \\
\{H_1, H_2, H_3\} & 1200 \le R_p < 1800 \\
\{H_1, H_2, H_3, H_4\} & 1800 \le R_p < 2200 \\
\mathcal{H} & R_p \ge 2200
\end{cases}
\]

This ensures computational scaling based on difficulty.

\section{Comparative Strategy Framework}

Let candidate strategies be:

\[
\mathcal{S} = \{S_1, S_2, ..., S_k\}
\]

Each strategy is evaluated by:

\[
E(S_i) = w_1 \cdot \text{complexity} + 
w_2 \cdot \text{runtime\_empirical} + 
w_3 \cdot \text{confidence}
\]

Final selection:

\[
S^* = \arg\min_{S_i \in \mathcal{S}} E(S_i)
\]

Subject to correctness constraints.

\section{Verification Framework}

Verification consists of:

\begin{enumerate}
    \item Sample test validation
    \item Randomized test generation
    \item Adversarial edge case synthesis
\end{enumerate}

Let:

\[
V = \bigwedge_{j=1}^{m} \text{Test}_j(A) = \text{pass}
\]

Only if $V = \text{true}$ may finalization occur.

\section{Chess Strategy Mapping Formalization}

Define mapping:

\[
\mathcal{M}: \mathcal{A} \rightarrow \mathcal{C}
\]

Where:
\begin{itemize}
    \item $\mathcal{A}$ = Algorithmic paradigms
    \item $\mathcal{C}$ = Chess strategic archetypes
\end{itemize}

Example mappings:

\[
\text{Greedy} \rightarrow \text{Initiative}
\]
\[
\text{Dynamic Programming} \rightarrow \text{Positional Accumulation}
\]
\[
\text{Game Theory} \rightarrow \text{Perfect Adversarial Play}
\]

This mapping is explanatory only and does not influence correctness.

\section{Scalability Analysis}

Worst-case multi-agent invocation cost:

\[
O(k \cdot T_{avg})
\]

Where:
\begin{itemize}
    \item $k$ = number of activated agents
    \item $T_{avg}$ = average token usage per agent
\end{itemize}

With adaptive activation:

\[
E[C_{total}] \ll C_{max}
\]

Because higher-tier agents are invoked infrequently.

\section{Failure Modes}

\begin{itemize}
    \item Reinforced internal hallucination
    \item Over-escalation on trivial problems
    \item Excessive token growth
    \item Premature convergence
\end{itemize}

Mitigation strategies include strict budgets, deterministic checks, and hard termination rules.

\section{Future Work}

\begin{itemize}
    \item Reinforcement learning for agent promotion/demotion
    \item Self-play evaluation against historical Codeforces data
    \item Empirical benchmarking across rating bands
    \item Distributed execution framework
\end{itemize}

\section{Experimental Evaluation Plan}

Evaluation metrics:

\begin{itemize}
    \item Solve rate by rating bucket
    \item Average token usage per rating
    \item Verification failure rate
    \item Escalation frequency
    \item Cost per solved problem
\end{itemize}

Benchmark datasets:
\begin{itemize}
    \item Historical Codeforces problem sets
    \item Random unseen problem batches
\end{itemize}

Statistical significance measured using solve-rate comparison against single-agent baseline.

\section{Orchestration Algorithm}

The hierarchical workflow is coordinated by a central orchestrator.

\subsection{High-Level Pseudocode}

\begin{verbatim}
function SOLVE_PROBLEM(problem_id):

    P <- FETCH_FROM_CODEFORCES(problem_id)
    INIT_SESSION_WORKSPACE(P)

    active_agents <- ACTIVATE_BY_RATING(P.rating)

    for agent in active_agents:
        M <- LOAD_RELEVANT_FILES(agent)
        response <- agent.PROCESS(P, M)
        UPDATE_WORKSPACE(response)

        if VERIFICATION_FAILED():
            ESCALATE()
        if CONFIDENCE_HIGH() and VERIFICATION_PASSED():
            break

    if VERIFICATION_PASSED():
        FINALIZE_SOLUTION()
        UPDATE_AGENT_RATINGS()
    else:
        MARK_UNRESOLVED()

    RESET_SESSION()
\end{verbatim}

\subsection{Escalation Rule}

Escalation is triggered if:

\[
\text{confidence}_i < \theta_i \quad \lor \quad V = \text{fail}
\]

Escalation depth is capped at $d_{max}$.

\section{Real-Time Codeforces Integration}

\subsection{API Endpoints}

The system interacts with the Codeforces API:

\begin{itemize}
    \item \texttt{problemset.problems}
    \item \texttt{contest.list}
    \item \texttt{contest.standings}
\end{itemize}

\subsection{Problem Fetch Model}

Let:

\[
P_{live} = (I, O, C, T, R_p)
\]

Where $R_p$ is problem rating.

The system fetches metadata and automatically classifies:

\[
\hat{A}_0 = f(T, C)
\]

Where $f$ is initial paradigm predictor.

\section{Deployment Architecture}

\subsection{Microservice Structure}

The system can be decomposed into:

\begin{itemize}
    \item Orchestrator Service
    \item Agent Pool Service
    \item Execution Sandbox
    \item Verification Engine
    \item Rating Manager
    \item Session Memory Manager
\end{itemize}

\subsection{Service Interaction Flow}

\begin{enumerate}
    \item Orchestrator receives problem request
    \item Session initialized
    \item Agent invoked
    \item Response stored in session memory
    \item Code executed in sandbox
    \item Verification result returned
    \item Decision updated
\end{enumerate}

\subsection{Isolation Constraints}

Each problem execution occurs in:
\begin{itemize}
    \item Isolated container
    \item CPU-time limited sandbox
    \item Memory bounded environment
\end{itemize}

\section{Complexity Bound on Multi-Agent Overhead}

Let:
\begin{itemize}
    \item $k$ = number of active agents
    \item $r$ = revision cycles
    \item $T_{max}$ = max tokens per agent
\end{itemize}

Worst-case token cost:

\[
C_{worst} = k \cdot r \cdot T_{max}
\]

Given bounded $k$ and $r$:

\[
C_{worst} = O(1)
\]

Relative to number of agents.

Thus, recursion depth does not grow unbounded.

\section{Convergence Guarantee}

Define a monotonic improvement function:

\[
\phi(\Sigma_i) = (\text{confidence}, V)
\]

Each escalation must satisfy:

\[
\phi(\Sigma_{i+1}) \ge \phi(\Sigma_i)
\]

Under partial ordering where:
\begin{itemize}
    \item Verification failure < Verification success
    \item Lower confidence < Higher confidence
\end{itemize}

If no improvement occurs within $r_{max}$ cycles, session terminates.

\section{Promotion and Demotion Mechanism}

Each agent $H_i$ has rating $R_i$.

Promotion rule:

\[
R_i > R_{threshold}^{i+1} \Rightarrow \text{eligible for promotion}
\]

Demotion rule:

\[
R_i < R_{threshold}^{i} \Rightarrow \text{restricted scope}
\]

This enables adaptive specialization over time.

\section{Confidence Calibration Model}

Let:
\[
\text{confidence} = g(\text{verification pass rate}, \text{complexity margin})
\]

Example:

\[
\text{confidence} =
0.5 \cdot \text{test\_pass\_ratio}
+
0.3 \cdot \text{complexity\_margin}
+
0.2 \cdot \text{historical\_accuracy}
\]

Confidence threshold $\tau$ determines finalization.

\section{Security and Integrity Considerations}

\begin{itemize}
    \item Prevent prompt injection via strict input parsing
    \item Validate API inputs
    \item Limit execution sandbox permissions
    \item Prevent cross-session contamination
\end{itemize}

\section{Distributed Scaling Model}

For large-scale deployment:

\begin{itemize}
    \item Horizontal scaling of agent pools
    \item Queue-based orchestration
    \item Async verification pipelines
    \item Token budget monitoring service
\end{itemize}

Load balancing ensures high-rating problems receive priority compute.

\section{Empirical Study Design}

To evaluate system robustness:

\subsection{Baseline Comparison}

Compare against:
\begin{itemize}
    \item Single-agent LLM baseline
    \item Deterministic template solver
\end{itemize}

Metrics:

\begin{itemize}
    \item Solve rate by rating bucket
    \item Average cost per problem
    \item Escalation depth frequency
    \item Hallucination detection rate
\end{itemize}

\subsection{Ablation Studies}

Remove one component at a time:

\begin{itemize}
    \item Remove hierarchy
    \item Remove verification
    \item Remove token cap
    \item Remove session isolation
\end{itemize}

Measure degradation.

\section{Theoretical Positioning}

This system lies at intersection of:

\begin{itemize}
    \item Multi-agent reasoning systems
    \item Structured memory architectures
    \item Cost-aware LLM orchestration
    \item Educational abstraction frameworks
\end{itemize}

It represents a hybrid symbolic-neural hierarchical solver model.

\end{document}